% Reaction Ref. List Template (Just copy the LaTeX generated table into this document in place of the currenttable and compile it using pdflatex etc. then convert it into html, then open it back into excel and 
% this makes a linkable references excel spreadsheet


\documentclass[a4paper, 12pt,epsfig, onecolumn]{article}
\pagestyle {plain}
%-------- packages to import for use ------------------
\usepackage{psfig}
\usepackage{graphics}
\usepackage{rotating}
\usepackage{pslatex}
\usepackage{mathptm}
\usepackage{makeidx}
\usepackage{pslatex}
\usepackage{moreverb,hangcaption}
\usepackage{longtable}
\usepackage{multicol}
\usepackage{subfigure}
\usepackage{comment}
%\usepackage{nature}
%\usepackage{naturefem}
%\usepackage{citesupernumber}



\usepackage{sfmath} 


\usepackage{html}
\usepackage{url} 

\usepackage[normal]{caption2}
\renewcommand{\tablename}{Supplementary Table}
\renewcommand{\captionlabelfont}{\bf}
\renewcommand{\captionlabeldelim}{ \textbf{$\vert$}}
\renewcommand{\captionmargin}{1cm} 


%-------------for dvi->ps -----------------------------
\usepackage{hyperref}
%\usepackage[dvips]{color}


\usepackage[section]{placeins}
\usepackage{colortbl}

%----------------- General Formatting -----------------------
\setlength{\textwidth}{14.5cm}     %setlength change the "distances"
%\setlength{\textheight}{24.0cm}
\setlength{\oddsidemargin}{0.6cm}
%\setlength{\evensidemargin}{0.6cm}
%\setlength{\topmargin}{-1.15cm}
%\setlength{\parindent}{1.5cm}
\pagenumbering{arabic}

%\newcommand{\mydot}[1]{\ensuremath{\hspace{-1mm} \cdot \hspace{-1mm}}
\newcommand{\mydot}{\hspace{-0mm} $\cdot$  \hspace{-0mm}}

% ----------begin latex input----------------------------------
\begin{document}

{\sf


\noindent
\textbf{Supplementary Table 2.} Hyperarcs of the logical T-cell signaling model (see  Fig. 1 and methods). Exclamation mark ('!') denotes a logical NOT and dots within the
  equations indicate AND operations. The names of the substances in the  explanations are those used in the model and Fig. 1; the biological names are displayed in the Supplementary Table 1. In the
  case where two pools of a molecule were considered (e.g. lckp1 and lckp2),
  a'reservoir' (lckr) was included which was required for both pools. This
  allows to perform a simultaneous knock-out of both pools acting on the reservoir.



\begin{center}
\begin{longtable}{|p{0.03\textwidth}|p{0.26\textwidth}|p{0.008\textwidth}|p{0.6\textwidth}|}
\hline

\textbf{Nr} & \textbf{Reaction} & \textbf{$\tau$} & \textbf{Documentation} \\ \endhead  \hline
1&       $\rightarrow$ cd28 &1& Binding of ligand or antibody to cd28 is an input of the model. \\ \hline
2&       $\rightarrow$ cd4  &1& Binding of ligand or antibody to cd4 is an input of the model. \\ \hline
3&     $\rightarrow$ tcrlig &1& Binding of ligand or antibody to the tcr is an input of the model. \\ \hline
4& !bad $\rightarrow$ bclxl               &1& bad inhibits bclxl\cite{zha:1997,yang:1995}. \\ \hline
5& !cabin1\mydot !calpr1\mydot !akap79\mydot cam $\rightarrow$        calcin  &1&  cam binds to and activates calcineurin (calcin), while  cabin1, calpr1, akap79 inhibit calcin\cite{Matsuda_EMBO_2000,ryeom:2003,Kashishian_JBC_1998}. \\ \hline
6& !camk4 $\rightarrow$ cabin1                              &1& camk4 regulates via phosphorylation nuclear export of Cabin1\cite{pan:2005}. \\ \hline
7& card11a \mydot pkcth $\rightarrow$ ikkg  &1& The complex card11+bcl10 +malt1 is required for ikkg activation \cite{Thome_NatRevImm_2004,weil:2006,MatthewS}. Phosphorylation, probably via pkcth\cite{khoshnan:2000}, is also required.\\ \hline
8& !ccblp1\mydot tcrlig $\rightarrow$ tcrb  &1& Binding        of ligand activates the tcr, while active ccbl ubiquinates it, thus        leading to tcr degradation\cite{Huang_JBC_04}. \\ \hline
9& !ccblp1\mydot tcrp\mydot abl $\rightarrow$ zap70            &1& abl phosphorylates and thus activates zap70\cite{ZipfelPA_CurrBiol_04} once it is bound to the tcr. Active ccbl  \phantom{\emph{}} can degrade zap70. \\ \hline
10& !cd28 $\rightarrow$ cblb               &2& cd28 induces cblb ubiquitination and degradation\cite{ZhangJ_JImm_02} after the early events thus, $\tau$=2. \\ \hline
11& !dgk\mydot plcga $\rightarrow$ dag &1& The active form of plc $\gamma$ 1 (plcga) splits pip2 into diacylglycerol (dag) and ip3 (see hyperarc 83)\cite{Huang_JBC_04}. Active dgks degrade dag into phosphatic acid\cite{TophamMK_JCB_06}. \\ \hline
12& !erk\mydot lckp1 $\rightarrow$ shp1                           &2& lck phosphorylates shp1 leading to its activation which allows it to dephosphorylate and thus deactivate lck. erk phosphorylates lck at p59, protecting it from shp1's effect\cite{StefanovaI_Nature_02,Altan-Bonnet_PLoSBiol_05}. Since shp1 activation comes some time after lck activation, it takes place at $\tau$=2. \\ \hline
13& !gab2 \mydot zap70 \mydot gads $\rightarrow$ slp76        &1& slp76 associates with lat via gads\cite{HorejsiV_NatRevImm_04,TogniM_MolImm_04}. gab2 competes for binding, and thus inhibits binding of slp76 to gads\cite{yamasaki:2001,yamasaki:2003}. \\ \hline
14& !gsk3 $\rightarrow$ bcat                              &1& Gsk3 inhibits bcat\cite{Liang_CC_03}. \\ \hline
15& !gsk3 $\rightarrow$ cyc1                       &1& Gsk3 inhibits cyc1\cite{Liang_CC_03}. \\ \hline
16& !ikb $\rightarrow$ nfkb                      &1& nfkb is retained in the cytoplasm by tight binding to the inhibitory protein ikb\cite{Huang_JBC_04}. \\ \hline
17& !ikkab $\rightarrow$ ikb               &1& ikb is phosphorylated by ikkab, leading to its ubiquination and subsequent degradation\cite{Huang_JBC_04,Krauss}. \\ \hline
18& ikkg\mydot camk2 $\rightarrow$ ikkab           &1&        Both the regulatory molecule ikkg and phosphorylation probably (but not only) via camk2 are required for the activation of the kinase subunits ikkalpha and beta (ikkab)\cite{Thome_NatRevImm_2004,weil:2006,MatthewS}. \\ \hline
19& !pkb $\rightarrow$ bad              &1& pkb inhibits bad\cite{hanada:2004}. \\ \hline
20& !pkb $\rightarrow$ fkhr             &1& pkb inhibits fkhr\cite{hanada:2004}. \\ \hline
21& !pkb $\rightarrow$ gsk3        &1& pkb inhibits gsk3 \cite{Liang_CC_03,hanada:2004}. \\ \hline
22& !pkb $\rightarrow$ p21c        &1& pkb inhibits p21c \cite{Liang_CC_03,hanada:2004}. \\ \hline
23& !pkb $\rightarrow$ p27k          &1& pkb inhibits p27k \cite{Liang_CC_03,hanada:2004}. \\ \hline
24& !gadd45\mydot zap70 $\rightarrow$ p38          &1&        gadd45 inhibits the zap70 mediated activation of p38\cite{SalvadorJM_NatImmunol_05}. \\ \hline
25& !shp1\mydot cd45\mydot cd4\mydot !csk\mydot lckr $\rightarrow$ lckp1  &1& Full activation of the cd4-bound pool (there is also a tcr-dependent pool, see hyperarc 62/64 and legend) of lck requires dephosphorylation of the negative regulatory site (by cd45, and in absence of csk, which phosphorylates it) and autophosphorylation of the positive regulatory site, which cd4-bound lck can perform upon cd4 crosslinking\cite{PalaciosEH_Onco_04}. \\ \hline
26& !tcrb $\rightarrow$ pag                               &1& upon ligand binding to the tcr, pag is dephosphorylated by an unidentified phosphatase (probably cd45)\cite{DavidsonD_MCB_03}. \\ \hline
27& ap1 $\rightarrow$                                     &1& the transcription factor ap1 is an output of the model. \\ \hline
28& bcat $\rightarrow$                            &1& bcat is an output of the model. \\ \hline
29& bclxl $\rightarrow$                                     &1& bclx is an output of the model. \\ \hline
30& ca $\rightarrow$ cam                    &1& calcium binds to calmodulin and this complex to calcineurin\cite{Feske_NI_2001}. \\ \hline
31& calcin $\rightarrow$ nfat                        &1& calcineurin dephosphorylates nfat leading to  nuclear translocation and activation of nfat\cite{Huang_JBC_04,macian:2005,Krauss}.\\ \hline
32& cam $\rightarrow$ camk4             &1& camk2 activation is dependent on calmodulin (cam)\cite{Anderson_Biometals_1998}. \\ \hline
33& ccblr\mydot fyn $\rightarrow$ ccblp2                  &2& Upon Fyn phosphorylation, ccbl can inhibit  plcg\cite{Rellahan_ECR_03}. This is one out of 2 mechanisms ccbl is        involved in, and we call it ccblp2 (pool 2, see legend)\phantom{\emph{}}. Since ccbl mediated inhibition is slower than the early events, $\tau$=2. \\ \hline
34& ccblr\mydot zap70 $\rightarrow$ ccblp1  &2& ccbl binds to activated (and thus phosphorylated) zap70, leading to the ubiquination and subsequent degradation of zap70 and tcr\cite{RaoN_JImm_00}.  This is one out of 2 mechanisms ccbl is involved in, and we call it ccblp1 (pool 1, see legend)\phantom{\emph{}}. Since ccbl mediated degradation has to be slower than the early events, $\tau$=2. \\ \hline
35& x $\rightarrow$ vav1                   &1& CD28 stimulation leads to Vav1 activation\cite{MichelF_Jimmunol_00,Raab2001,HehnerSP_JBC_00,Kovacs2005}, a process mediated by a yet unidentified kinase (see hyperarc 48). \\ \hline
36& cdc42 $\rightarrow$ mekk1              &1& The GTP         bound cdc42 (and rac1, see hyperarc 87) is able to bind mekk1\cite{FangerGR_EMBO_1997}; CD28 activates mekk1  in a  cdc42 mediated manner\cite{Kaga_JI_1998}. \\ \hline
37& cre $\rightarrow$                            &1& cre is an output of the model. \\ \hline
38& creb $\rightarrow$ cre                         &1& The creb protein is a transcription factor that binds to cre ativating the related genes\cite{Krauss}. \\ \hline
39& cyc1 $\rightarrow$                               &1& cyc1 is an output of the model, and is involved in  cell cycle regulation\cite{Krauss}. \\ \hline
40& dag $\rightarrow$ rasgrp  &1& dag causes the cytoplasmic rasgrp1 to move to the golgi, where it can act on Golgi associated-ras\cite{DiFiore_Nature_03,BivonaTG_Nature_03}. Even though pkcth phosphorylates rasgrp at t184,\cite{RooseJP_MCB_05} we did not include  connection pkcth  $\rightarrow$  rasgrp1 since this effect is not specific to pkcth, but general to other pkcs (less well-characterized in T cells and therefore not included in the model); inclusion of this effect would make this step strictly dependent on pkcth, which is not the case. \\ \hline
41& dag\mydot vav1\mydot pdk1 $\rightarrow$ pkcth                 &1& Activation of pkcth requires binding to dag, phosphorylation by pdk1\cite{LeeKY_Science_05}, and vav1\cite{VillalbaM_JCB_02}. \\ \hline
42& erk $\rightarrow$ fos                                 &1& erk phosphorylates fos\cite{Huang_JBC_04}. \\ \hline
43& erk $\rightarrow$ rsk                         &1& erk activates rsk via phosphorylation\cite{FrodinM_MCE_99}. \\ \hline
44& fkhr $\rightarrow$                            &1& The transcription factor fkhr is an output of the model. \\ \hline
45& fos\mydot jun $\rightarrow$ ap1                       &1& Binding of jun with fos leads to the formation of ap1\cite{Huang_JBC_04,Krauss}. \\ \hline
46& fyn $\rightarrow$ abl     &1& abl kinases are activated following tcr stimulation via a Src kinase (lck or fyn, see hyperarc 59)\cite{ZipfelPA_CurrBiol_04}. \\ \hline
47& fyn $\rightarrow$ pag                               &2& fyn phosphorylates pag\cite{DavidsonD_MCB_03}, leading to the binding of csk. This process takes place 3-5 min after tcr activation, and thus it belongs to the time scale $\tau$=2\cite{TorgersenKM_JBC_01}. \\ \hline 
48& cd28 $\rightarrow$ x                                &1& Vav1 activation requires cd28 activation\cite{MichelF_Jimmunol_00,Raab2001,Kovacs2005} and is mediated by an non-identified kinase x (see hyperarcs 35 and 63). \\ \hline
49& gab2 $\rightarrow$ shp2                                &1& Gab2 recruits shp2\cite{arnaud:2004}. \\ \hline
50& gads\mydot lat\mydot zap70 $\rightarrow$ gab2                   &2& zap70 phosphorylates gab2 upon binding to lat and gads\cite{yamasaki:2001,yamasaki:2003}.  This process must take place after the early events to allow signal propagation, thus $\tau$=2. \\ \hline
51& grb2\mydot lat\mydot zap70 $\rightarrow$ gab2                  &2& zap70 phosphorylates gab2 upon binding to lat and grb2\cite{yamasaki:2001,yamasaki:2003}. This process must take place after the early events to allow signal propagation, thus $\tau$=2. \\ \hline
52& hpk1 $\rightarrow$ mekk1                           &1& hpk1 binds and phosphorylates mekk1\cite{HuMC_GDev_96}. \\ \hline
53& hpk1 $\rightarrow$ mlk3                            &1& hpk1 binds and phosphorylates mlk3\cite{Tibbles_EMBO_1996}. \\ \hline
54& ip3 $\rightarrow$ ca                                 &1& Binding of ip3 to the ip3 receptor in the endoplasmatic reticulum leads to the release of calcium\cite{Chan_ARB_1999}. \\ \hline
55& jnk $\rightarrow$ jun                                &1& jnk phosphorylates jun\cite{Krauss}. \\ \hline
56& lat $\rightarrow$ grb2                     &1& grb2        (which in turn binds sos) can bind to phosphorylated lat\cite{LindquistJA_ImmRev_03}\cite{HorejsiV_NatRevImm_04}. \\ \hline
57& lat $\rightarrow$ hpk1                              &1& hpk1 binds to lat and is recruited to the lipid raftss\cite{LiouJ_Immunity_00}. \\ \hline
58& lat $\rightarrow$ plcgb                                &1& plcgamma binds to lat\cite{HorejsiV_NatRevImm_04,TogniM_MolImm_04}. \\ \hline
59& lckp1 $\rightarrow$ abl            &1& abl kinases        are activated following tcr stimulation via a Src kinase (lck or fyn,        see hyperarc 46)\cite{ZipfelPA_CurrBiol_04}. \\ \hline
60& lckp1 $\rightarrow$ rlk                                &1& lck phosphorylates rlk leading to its activation\cite{ShanX_MCB_00}. \\ \hline
61& lckp1\mydot cd45 $\rightarrow$ fyn                       &1& lck activates fyn\cite{FilippD_JImm_04}, a process where the        dephosphorylation of the negative regulatory site of fyn by cd45 is also required. \\ \hline
62& lckp2\mydot !cblb $\rightarrow$ pi3k                     &1& pi3k is dependent on the Src kinase lck for activation\cite{DeaneJA_AnnRevImmunol_04}. Additionally, cblb promotes pi3k ubiquination\cite{FangD_NatImmunol_01}. \\ \hline
63& x \mydot !cblb $\rightarrow$ pi3k                      &1&  pi3k is also activated upon CD28\cite{august:1994,ghiotto_ragueneau:1996}  via an non-determined kinase x (see hyperarc 48). Even though Lck has been proposed to be involved in this process\cite{vonWillebrandM_EJI_94,GibsonS_BiochemJ_98,holdorf:1999,huang:2002}, our experiments show that, at least for primary human T-cells, PI3K activation is not strictly Src-kinase dependent (see Fig. S4). A reasonable candidate would be a Tec kinase, but since it is not experimentally verified, we keep an undetermined x.  \\ \hline 
64& lckr\mydot tcrb $\rightarrow$ lckp2 &1& The activation        of pi3k is determined by a second pool of lck (lckp2) (see legend)\phantom{\emph{}}  which can be activated by tcr activation\cite{OkkenhaugK_NatRevImmunol_03}. \\ \hline
65& malt1 \mydot card11 \mydot  bcl10 $\rightarrow$ card11a                &1& The binding of  malt1 to card11 and bcl10 forms the active card11 complex\cite{Thome_NatRevImm_2004,Gaide_NI_2002, Uren_MolCell_2000, Lucas_JBC_2001}. \\ \hline
66& mek $\rightarrow$ erk                      &1& mek phosphorylates erk  leading to erk activation\cite{Huang_JBC_04,Krauss}. \\ \hline
67& mekk1 $\rightarrow$ jnk                  &1& mekk1 activates jnk\cite{Davis_Cell_2000}. \\ \hline
68& mekk1 $\rightarrow$ mkk4             &1& mekk1 is able  to phosphorylate MKK4 leading to its activation\cite{yan:1994}. \\ \hline
69& mekk1 $\rightarrow$ p38                            &1&   mekk1 leads to p38 activation\cite{Guan_JBC_1998}. \\ \hline
70& mkk4 $\rightarrow$ jnk                                  &1&  MKK 4 activates jnk\cite{Davis_Cell_2000,Derijard_Science_1995}. \\ \hline
71& mlk3 $\rightarrow$ mkk4                         &1& mlk3 phosphorylates mkk4\cite{Tibbles_EMBO_1996}. \\ \hline
72& nfkb $\rightarrow$                                 &1& nfkb is an output of the model. \\ \hline
73& p21c $\rightarrow$                &1& p21cip is an output of the model controlling the cell cycle. \\ \hline
74& p27k $\rightarrow$                    &1& p27kip is an output of the model controlling the cell cycle. \\ \hline
75& p38 $\rightarrow$                  &1& p38 is an output of the model. \\ \hline
76& p70s $\rightarrow$         &1& p70s is an output of the model. \\ \hline
77& pag $\rightarrow$ csk        &1& Phosphorylation of pag allows csk to bind it and then act on lck\cite{LindquistJA_ImmRev_03,HorejsiV_NatRevImm_04}.  \\ \hline
78& pdk1 $\rightarrow$ p70s      &1& pkd1 phosphorylates p70s leading to its activation\cite{pullen:1998,hinton:2004}. \\ \hline
79& pdk1 $\rightarrow$ pkb                    &1& pdk1 phosphorylates pkb leading to its activation\cite{LafontV_FEBSletters_00,alessi:1997,anderson:1998}. \\ \hline
80& pi3k\mydot !ship1\mydot !pten $\rightarrow$ pip3    &1& pi3k leads to the production of pip3, while ship1 and pten inhibit this process\cite{okkenhaug:2004,Rameh99}. \\ \hline
81& pip3 $\rightarrow$ pdk1          &1& pip3 is required for pdk1 activation\cite{ mora:2004}. \\ \hline
82& pip3\mydot zap70\mydot slp76 $\rightarrow$ itk    &1& When phosphorylated, slp76 can bind to itk; additional binding to pip3 and phosphorylation via zap70 activates itk\cite{Huang_JBC_04,TogniM_MolImm_04,CzarMJ_BiochemSocTran_01}. \\ \hline
83& plcga $\rightarrow$ ip3                &1& Active plcga splits pip2 into ip3 and diacylglycerol (dag,see hyperarc 11)\cite{Huang_JBC_04,TogniM_MolImm_04}. \\ \hline
84& plcgb\mydot !ccblp2\mydot slp76\mydot zap70\mydot vav1\mydot itk $\rightarrow$ plcga  &1& Once bound to phosphorylated lat, plcgb is activated by the combined action of vav and itk (or rlk, see hyperarc 85)\cite{CzarMJ_BiochemSocTran_01}. Additionally, binding to slp76 (phosphorylated by zap70)  is required to establish and stabilize the complex. Activated ccbl degrades plcga\cite{Rellahan_ECR_03}. \\ \hline
85& plcgb\mydot !ccblp2\mydot zap70\mydot vav1\mydot slp76\mydot rlk        $\rightarrow$ plcga  &1& Once bound to phosphorylated lat,        plcgb is activated by the combined action of vav and rlk (or itk, see hyperarc 84)\cite{CzarMJ_BiochemSocTran_01}. Additionally, binding to slp76 (phosphorylated by zap70)  is required to establish and stabilize the complex. Activated ccbl degrades plcga\cite{Rellahan_ECR_03}. \\ \hline
86& rac1p1 $\rightarrow$  mlk3   &1& Rac1p1 activates mlk3\cite{teramoto:1996}. \\ \hline
87& rac1p2 $\rightarrow$ mekk1     &1& GTP-bound Rac1p2        \phantom{\emph{}}  is able to bind mekk1\cite{FangerGR_EMBO_1997}, and active mekk1 leads to JNK activation\cite{minden:1995, Kaga_JI_1998}. \\ \hline
88& rac1p2 $\rightarrow$ sre   &1& Vav3-dependent Rac1 is able to activate Sre via SRF\cite{hill:1995}. \\ \hline
89& rac1r\mydot vav1 $\rightarrow$ rac1p1 &1& Downregulation of Vav1 but not Vav3 affects IL-2 production in T cells\cite{zakaria:2004} via the rac1-mediated jnk pathway.  Since rac1 mediates this process, we defined a vav1-dependent pool of rac1 (see hyperarc 90 and legend).\\ \hline
90& rac1r\mydot vav3 $\rightarrow$ rac1p2          &1& Downregulation of Vav3 but not Vav1 affects Sre activity\cite{zakaria:2004}. Since rac1 mediates this process, we defined a vav3-dependent pool of rac1 (see hyperarc 89 and legend)\\ \hline
91& raf $\rightarrow$ mek                              &1& Raf phosphorylates mek  leading to mek activation\cite{FranklinRA_JClinInvest_94}. \\ \hline
92& ras $\rightarrow$ raf                                  &1& Ras mediates raf localization to the membrane, and consequently, raf is activated\cite{Krauss}.\\ \hline
93& rsk $\rightarrow$ creb   &1& Rsk phosphorylates creb increasing its activity\cite{FrodinM_MCE_99}. \\ \hline
94& sh3bp2 $\rightarrow$ vav3        &1& sh3bp2 binds vav3 via an sh2 domain, leading to its activation\cite{zakaria:2004}. \\ \hline
95& sos\mydot !gap\mydot rasgrp $\rightarrow$ ras  &1& Bound to lat via grb2, sos catalyzes the exchange of GTP for GDP in the cellular-membrane-located ras, while rasgrp1 catalyzes the exchange of GTP for GDP in golgi-located ras\cite{DiFiore_Nature_03}. In turn gap catalyzes the conversion GTP to GDP and thus deactivates ras\cite{GenotE_CurrOpinImmunol_00}. \\ \hline
96& sre $\rightarrow$                                  &1& Sre is an output of the model. \\ \hline
97& tcrb $\rightarrow$ dgk                    &2& dgks get activated after tcr activation in yet an unclear manner, we therefore make it dependent on activation of the tcr. Since dag must be produced in the early events, we assign it a $\tau$=2\cite{SanjuanMA_JImmunol_03}. \\ \hline
98& tcrb\mydot fyn $\rightarrow$ tcrp             &1& Upon ligand binding to the tcr, active fyn can phosphorylate the tcr\cite{FilippD_MolImm_04}. \\ \hline
99& tcrb\mydot lckp1 $\rightarrow$ tcrp               &1&        The  co-localization of tcr with cd4 mediated by peptide-MHC or        antibody crosslinking results in an increased local concentration of lck around the tcr leading to phosphorylation of ITAMs\cite{FilippD_JImm_04}. \\ \hline
100& tcrb\mydot lckr $\rightarrow$ fyn                           &1& A fraction of fyn is bound to the tcr, and tcr crosslinking  leads to fyn auotophosphorylation and activation\cite{FilippD_MolImm_04}. Since lck is required in the development for having capable fyn\cite{ZamoyskaR_ImmunolRev_03}, lckr (existence of lck in the cell) is required as well. \\ \hline
101& zap70 $\rightarrow$ lat                             &1& zap70 phosphorylates lat at different sites\cite{Huang_JBC_04}. \\ \hline
102& zap70\mydot lat $\rightarrow$ sh3bp2                      &1& sh3bp2 binds to phosphorylated lat upon phosphorylation by zap70\cite{Qu_Biochem_05}. \\ \hline
103& zap70\mydot sh3bp2 $\rightarrow$ vav1             &1& zap70 phosphorylates vav1\cite{zakaria:2004} which together with binding of vav1 to sh3bp2\cite{Qu_Biochem_05}, leads to vav1 activation. \\ \hline
104&$\rightarrow$ card11 &1& Regulation of card11 is not clear, thus we set an external input to it. Default value is 1.\\ \hline
105&$\rightarrow$ gadd45 &1& Regulation of  gadd45 is not clear, thus we set an external input to it. Default value is 1.\\ \hline
106&$\rightarrow$ gap &1& GTP activating proteins (gaps) are important regulators of ras activation but their own regulation is not clear\cite{CantrellDA_ImmRev_03}. Therefore they are included in the model with an external input. \\ \hline
107&$\rightarrow$ lckr &1& Input to the system (presence of Lck in the cell). Default value is 1. \\ \hline
108 & cam $\rightarrow$ camk2 &1&  cam (calmodulin) activates calmodulin-dependent kinase II (camk2) \cite{Hughes_JBC_2001}. \\ \hline
109 & grb2 $\rightarrow$ sos &1& sos binds to grb2 and thus get recruited to the membrane via lat\cite{BudayL_JBC_94}. \\ \hline
110& lat $\rightarrow$ gads  &1& gads can bind to phosphorylated lat\cite{TogniM_MolImm_04,HorejsiV_NatRevImm_04}. \\ \hline
111& cdc42 $\rightarrow$ sre   &1& cdc42 is able to activate Sre via SRF\cite{hill:1995}. \\ \hline
112 & nfat $\rightarrow$  &1& nfat is an output of the model. \\ \hline
113 & shp2 $\rightarrow$  &1& shp2 is an output of the        model.  \\ \hline
114&       $\rightarrow$ cd45 &1& Regulation of cd45 is not clear, thus we set an external input to it. Default value is 1.\\ \hline
115&       $\rightarrow$ pten &1& Regulation of pten is not clear, thus we set an external input. Default value is 0.\\ \hline
116&      $\rightarrow$ bcl10 &1& Regulation of bcl10 is not clear, thus we set an external input to it. Default value is 1.\\ \hline
117&      $\rightarrow$ ccblr &1& Input to the system (presence of ccbl in the cell). Default value is 1.\\ \hline
118&      $\rightarrow$ cdc42 &1& Regulation of  cdc42 is not clear, thus we set an external input to it. Default value is 0.\\ \hline
119&      $\rightarrow$ malt1 &1& Regulation of malt1 is not clear, thus we set an external input to it. Default value is 1.\\ \hline
120&      $\rightarrow$ rac1r &1& Input to the system (presence of rac1 in the cell). Default value is 1.\\ \hline
121&      $\rightarrow$ ship1 &1& Regulation of ship1 is not clear, thus we set an external input. Default value is 0.\\ \hline
122&     $\rightarrow$ akap79 &1& Regulation of akap79 is not clear, thus we set an external input to it. Default value is 0.\\ \hline
123&     $\rightarrow$ calpr1 &1& Regulation of calpr1 is not clear, thus we set an external input to it. Default value is 0.\\ \hline


\end{longtable}
\end{center}


}

\renewcommand{\familydefault}{\rmdefault} 

\newpage
%\bibliographystyle{/home/saezr/Repository/latex/naturemagnourl}
\bibliographystyle{plos}
\bibliography{/home/saezr/projects/struct_an/doc/tcr_documentation/tcr_bibs/tcr_bib}



\end{document}  























